 % Преамбула
\documentclass[12pt, oneside, dvipsnames]{extarticle}
\input{pr_preamble}
\usepackage{cleveref}
\usepackage[
backend=biber,
style=alphabetic,
sorting=ynt
]{biblatex}
\addbibresource{mybib.bib}

\begin{document}
	
	\begin{center}
	{ \Huge \bf{Древесные симметрии} }

	\section*{Проекты}
	\end{center}

	\tableofcontents

	\begin{center}
		\subsection*{Стабилизация лучей и параболические подгруппы}
		\addcontentsline{toc}{subsection}{\protect\numberline{}Стабилизация лучей и параболические подгруппы}%
	\end{center}

	\begin{definition} 
		Пусть $X^*$~--- регулярное корневое дерево, $e$ луч, а $G \le \Aut(X^*)$~--- произвольная подгруппа. Тогда \emph{ассоциированная с $G$ параболическая подгруппа  $P_{e}$}~--- это $\Stab_{G}(e)$. 
	\end{definition}

	Иными словами, параболические подгруппы~--- это подгруппы, стабилизирующие лучи в $X^*$ (или же, точки $\partial X$). 

	Базовые (и несложные) факты про параболические подгруппы изложены в~\cite[стр. 8-9]{bartholdi2001parabolic} (всего несколько страничек).


	Введём несколько важных для нас групп, действующих на корневых деревьях. 

	\begin{definition} 
		Пусть $p$~--- простое число, $T_{p}$~--- $p$-регулярное корневое дерево. Группой \bf{$p$-базилики $\fB_{p}$} называется подгруппа в $\Aut(T_{p})$, порожденная элементами 
		 $$ a = (1,1, \ldots ,1,b), \quad b = (1,1, \ldots ,1,a)\sigma, \quad \text{ где } \sigma = (1 2  3 \ldots p) \in S_{p} $$

		 Прочитать подробнее про неё можно в~\cite{didomenico2021pbasilica}.
	\end{definition}

	\begin{definition} 
		\bf{$\mathbf{GGS}$-группой (или же группой Григорчука-Гупты-Сидки)} называется, подгруппа $\langle a, \sigma \rangle$ в $\Aut(T_{p})$, порожденная циклической перестановкой $\sigma = (123 \ldots p)$  и элементом 
		$$ a = (\sigma^{e_1},\sigma^{e_2},...,\sigma^{e_{p-1}},a), $$
		где $e_i \in \F_{p} = \Z/p\Z$.
	\end{definition}

	Подробнее о $\mathbf{GGS}$-группах и связанных с ними результатах про рост вы можете прочитать в работе~\cite{didomenico2022ggsgroups}. 

	\begin{definition} 
		Группа \bf{Гупты-Фабриковского} определяется следующим образом. Рассмотрим $A = C_{p} = \langle a \vert a^p = 1 \rangle $  и $p$-регулярное корневое дерево $A^* = T_p$.  Заметим, что тогда $a$ мы можем отождествить с циклической перестановкой $\sigma = (1 2 p)$, действующей на первом уровне. Определим автоморфизм $t \in \Aut(A^*) \cong \Aut(A^*) \wr S_{p} $, как $t = (a, 1, t)$. Заметим, что элементы $a$ и $t$ имеют порядок $p$. Группа $\mathbf{\Gamma} = \langle a, t \rangle \le \Aut(A^*)$ и будет называться группой \bf{Гупты-Фабриковского} порядка $p$. 
	\end{definition}

	Подробнее об этой группе и связанных с ней результатах про рост вы можете прочитать в работе Л. Бартольди~\cite{Bartholdi_2009}. 

	Кроме них, все мы помним про \bf{группу Григорчука $\mathbf{G}$}. 

	Все эти группы отличаются красотой и незаурядностью в их действии на деревьях и в связи с этим представляют видимый научный интерес. Про них предлагаются такие задачи: 

	\begin{enumerate}
		\item Для каждой группы посчитать параболические группы каких-то бесконечных лучей и изучить факторы по ним, возможно их рост.

		\item Правда ли, что елсли отфакторизовать группу 3-Базилики $\fB_{3}$ по параболической подгруппе $P_{e} = \Stab_{fB_{3}}(e)$, стабилизирующий некоторый луч в 3-регулярном дереве, группа $3$-базилики превратится в группу 2-базилики. И, если нет, то каков результат факторизации? 
	\end{enumerate}




	\begin{center}
		\subsection*{О модификации групп}
		\addcontentsline{toc}{subsection}{\protect\numberline{}О модификации групп}%
	\end{center}

	Понятие регулярного корневого дерева $T_{d}$ можно естественно обобщить следующим образом: 

	\begin{definition} 
		Пусть $\overline{m} = \{ m_n \}, m_n \ge 2$~--- последовательность натуральных чисел. \emph{Сферически однородным корневым деревом} $T_{\overline{m}}$ мы будем называть корневое дерево, имеюще такой вид: 	
		\begin{itemize}
			\item У него есть корень $\varnothing$, а также 
			\item $m_1$ вершин первого уровня, $m_1 m_2$ вершин второго уровня, \ldots, $m_1 m_2 \ldots m_n$ вершин $n$-го уровня ($n \in \N$).
			\item Каждая вершина уровня $n$ имеет $m_{n + 1}$ детей, расположенных на следующем уровне.
		\end{itemize}
	\end{definition}	

	\begin{figure}[tbph]
		\centering
		\includegraphics[scale = 0.5]{pictures/pic_3.png}
		\caption{Сферически однородное корневое дерево}
		\label{fig:pic_3.png}
	\end{figure}

	Более подробную информацию о них можно прочитать в работе Р.И. Григорчука~\cite[стр. 80 и далее]{Grig_Sperical}. 

	Так как эти деревья естественно обобщают $T_{d}$, возникает вопрос о нетривиальных вложениях $ G  \hookrightarrow \Aut(T_{\overline{m}})$ для различных подгрупп $G \le \Aut(T_{d})$. В связи с этим предлагаются следующие задачи: 

	\begin{enumerate}
		\item Предлагается изучить способы вложения групп $p$-базилики $\fB_{p}$, Гупты-Фабриковского, $\mathbf{GGS}$-групп в $\Aut(T_{\overline{m}})$. Возможно, вычислить количество вложений. 

		\item Попытаться  вложиьб группу 2-Базилики и группу Григорчука $\mathbf{G}$ в дерево с последовательностью слоев $(23)^\infty$.
	\end{enumerate}


	\begin{center}
		\subsection*{Геометрический проект}
		\addcontentsline{toc}{subsection}{\protect\numberline{}Геометрический проект}%
	\end{center}

	Существует понятие \bf{константы гиперболичности пространства}. Это число $\delta$, ассоциированное с ним и отвечающее тому, насколько его геометрия похожа на гиперболическую. 

	К сожалению этот инвариант работает так, что если $\delta < \infty$, то пространство "очень" гиперболическое, а если $\delta = \infty$, то всё, что мы знаем это то, что оно другое. То есть разделение очень дискретное, а хотелось бы критерия в духе "чем больше число тем более гиперболическое пространство".

	Чтобы избавиться от такой большой критичности, можно по каждому пространству $M$ определить функцию $\delta_M$, которая будет исполнять ровно такую роль. Это очень \bf{новый} инвариант и про него непонятно вообще ничего. 

	\noindent\bf{Предлагается две задачи:}
	\begin{enumerate}
		\item Построить примеры метрических пространств, у которых эта функция будет устроена определенным образом (заранее заданным).
		\item Вычислить эту функцию для конкретного пространства --- графа определенного вида. 
	\end{enumerate}

	\emph{Никаких конкретных пререквезитов не предполагается.}

	\printbibliography

\end{document}