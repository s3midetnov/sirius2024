\subsection{Группы}

	И вот мы наконец плавно подошли к одному из главных определений в нашем курсе. 

	\begin{definition} 
		Множество $G$ с заданной на нём бинарной операцией $\circ$ (мы часто будем называть её умножением) называется \emph{группой}, если 
		\begin{enumerate}
			\item $(a \circ b) \circ c = a \circ (b \circ c) = a \circ b \circ c$. 
			\item Существует \emph{нейтральный} элемент $e \in G$, то есть такой элемент, что $e \circ g = g \circ e = g$ для всех элементов $g \in G$. 
			\item У всех элементов $g \in G$ есть обратный элемент $g^{-1}$, то есть такой, что $g \circ g^{-1} = g^{-1} \circ g = e$.
		\end{enumerate}
	\end{definition}

	\begin{remark}
		Заметим, что любая группа преобразований является группой. Группы бывают разные, но в дальнейшем, группы преобразований будут основным примером групп для нас. 
	\end{remark}

	\begin{example}
		Вновь глупые числовые примеры. Пояснение примеров из прошлого параграфа. 
	\end{example}

	\begin{remark}
		Вот в этом месте я бы хотел сделать такой разгон про то, зачем это всё на самом деле (зачем переходить от изучения \emph{конкретных преобразований} к изучению \emph{групп, как обще алгебраических структур}. Хороший пример такого разгона в первых пяти минутах \href{https://www.youtube.com/watch?v=NKZ_2EzHaSI}{этой лекции}, гляньте пж. Мне кажется, что это \bf{очень важно} при изложении групп детям. 
	\end{remark} 

	После того, как мы вдоволь насладились примерами, можно начать изучать наши объекты с общей точки зрения. 

	\begin{observation} 
		Нейтральный элемент группы единственен. Обратный элемент к элементу группы $g \in G$ единственен. 
	\end{observation}

	\begin{remark}
		Какими аксиомами группы мы пользовались для доказательства? 
	\end{remark}

	\begin{definition} 
		Два элемента группы $a, b \in G$ называются \emph{коммутирующими} (или, перестановочными), если $ab = ba$. Если все элементы группы $G$ коммутируют между собой, $G$ называтеся \emph{Абелевой} или \emph{коммутативной}. 
	\end{definition}

	\begin{remark}
		Заметим, что \bf{не абелевы} группы бывают. Например, рассмотренная нами ранее группа симметрий треугольника $S_{3} \cong D_{3}$.
	\end{remark}

	\begin{example}
		Заметим, что $(ab)^{1} = b^{-1}a^{-1}$ (\emph{Сначала мы надеваем носок, а потом ботинок. С другой  стороны, сначала мы снимаем ботинок, а потом носок}).

		В качестве проверки понимания можно задать слушателям такой вопрос:  (\emph{очень уж много носков}) Пусть $a_1, a_2, \ldots, a_n$~--- элементы некоторой группы $A$. Какой элемент будет обратным к $a_1 a_2 \ldots a_n$?
	\end{example}

	\begin{observation}
		В группе можно сокращать равенство справа и слева. 
	\end{observation}


	\subsection{Группа перестановок}

	Мы об этом уже не раз говорили, теперь дадим формальное определение

	\begin{definition} 
		Пусть $X$~--- множество. Множество взаимно однозначных отображений $X \to X$ образует группу (отностельно композиции), её мы будем называть \emph{симметрической группой} на множестве $X$ и обозначать $S_{X}$. 
	\end{definition}

	\begin{remark}
		Если множество $X$ конечно и в нём $n$ элементов, то мы можем думать про него, как про множество $\{ 1, 2, \ldots, n \}$.  Тогда перестановки можно записывать в виде табличек вида 
		\[
			\sigma = \begin{pmatrix} 1 & 2 & \ldots & n \\ i_{1} & i_{2} & \ldots & i_{n} \end{pmatrix}.
		\]

		Симметрическу группу множества $\{ 1, 2, \ldots, n \}$ мы будем обозначать, как $S_{n}$.
	\end{remark}

	\textcolor{red}{\bf{Про группу перестановок вообще много чего можно говорить. Чего мы хотим про неё говорить?}}

	\subsection{Циклические группы, порядок элемента}

	\begin{definition} 
		Порядком элемента $g$ группы $G$ называют наименьшее натуральное $n$, для которого $g^n = e$. Если такого $n$ не существует, то говорят, что $g$ имеет бесконечный порядок. 

		Порядок элемента $g$ мы будем обозначать, как $\ord{g}$.
	\end{definition}

	\begin{example}
		Например, порядок элемнта $2$ в группе $\Z/10$ равен 5. А вот в группе $\Z$ любой ненулевой элемент имеет бесконечный порядок. 
	\end{example} 

	\begin{example}
		Рассмотрим правильный $n$-угольник и все повороты плоскости, переводящие его в себя. Заметим, что если мы возьмём поворот $R = R_{\frac{2\pi}{n}}$, то 

		\[
			R^n = R_{2\pi} = e,\  R^{n + 1} = R, \ R^{n + 2} = R^2 
		\]
		и так мы получим в точности все повороты, оставляющие $n$-угольник на месте. 
	\end{example}

	\begin{definition} 
		Группу $G$, состоящую из элементов $e, a^1, \ldots, a^{n - 1}$ (где элемент $a$ имеет порядок $n$) называют \emph{циклической группой порядка $n$, порожденной элементом $a$} (а обозначать её мы будем, как $C_n$). Элемент $a$ называется \emph{образующей} этой группы. 
	\end{definition}

	Как видно из примера выше, повороты правильного $n$-угольника образуют циклическую группу порядка $n$.

	\begin{remark}
		А еще бывает \emph{бесконечная циклическая группа}, это $\Z$ (а образующая у неё $1$). Про неё логично думать именно так, так как любой её элемент можно представить в виде $n \cdot 1$, где $n$~--- целое число.
	\end{remark}

	\subsection{Гомоморфизм, изоморфизм}

	Мы уже видели, что между многими группами можно построить теоретико-множественную биекцию, но ведь это не говорит нам, что эти группы одинаковые. 

	\begin{example}
		Ну, например есть группа $\Z/6$ и группа $S_{3}$. Так как обе группы имеют порядок 6, между ними легко построить теоретико-множественную биекцию, но видно, что группы на самом деле существенно разные (можно привести какой-то пример, где элементы разного порядка, или что-то в таком духе). 
	\end{example}

	\begin{definition} 
		Отображение $f$ между группами $(G, \circ)$ и $(H, *)$ называется \emph{гомоморфизмом}, если 
		\[
			\forall a, b \in G \quad  f(a \circ b) = f(a) * f(b).
		\]
	\end{definition}

	Условие в определении гомоморфизм говорит о \emph{сохранении стркутуры}. 

	\begin{definition} 
		Пусть $\varphi\colon G \to H$~--- гомоморфизм групп. Его \emph{ядром} называют $\varphi^{-1}(e_{H}) \subset G$. Иными словами, 
		\[
			\Ker{\varphi} \eqdef \{ g \in G \ \vert\  \varphi(g) = e_{H} \} = \varphi^{-1}(e_{H}).
		\]
		\emph{Образом} $\varphi$ называют его образ, как функции, то есть 
		\[
			\mathrm{Im}{\varphi} = \{ \varphi(g) \ \vert\  g \in G \} \susbet G
		\]
	\end{definition}

	\subsection{Подгруппы, порождение}

	\begin{definition} 
		 Пусть $S$ – подмножество группы $G$. Подгруппой, порожденной множеством $S$, называется наименьшая подгруппа в $G$, содержащая $S$.  Её мы будем обозначать, как $\langle S \rangle$.
	\end{definition}

	\begin{definition} 
		Пусть $G$~--- группа, $S$~--- подмножество. Подгруппа, порожденная $S$~--- это наименьшая подгруппа в $G$	
	\end{definition}

	



	

	






	

	




