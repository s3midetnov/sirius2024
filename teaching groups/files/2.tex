	
	\subsection{Действие группы на множестве}

	\begin{definition} 
		Пусть $G$~--- группа, а $X$~--- множество. Будем говорить, что $G$ действует на $X$ и писать $G \acts X$, если задана операция $G \times X \to X$ (образ пары $(g, x)$обозначается обычно просто $gx$), обладающая для любого $x \in X$ и $g, h \in G$ следующими свойствами:

		\begin{enumerate}
			\item $g(hx) = (gh)x$ (внешняя ассоциативность);
			\item $1 \cdot x = x$ (унитальность).
		\end{enumerate}
	\end{definition}

	\begin{example}
		Рассмотрим много примеров действий, чтобы понять, что же всё-таки происходит. 

		\begin{enumerate}
			\item Группа $S_n$ естественно действует на $\{ 1, \ldots, n \}$. 
			\item Гомоморфизм $\theta \colon G \to S_{X}$ задаёт действие по правилу
			\[
				gx = \theta(g)(x).
			\]
			\item $\mathrm{GL}_{n}(V) \acts V$. 

			\item Пусть $G = (\R^2, +)$. Зафиксируем векотр $v$ и рассмотрим действие $G \acts \R^2$, заданное по правилу $x \mapsto x + v$.
		\end{enumerate}
	\end{example}

	Теперь заметим, что действие само по себе естественным образом задаёт гомоморфизм $G \to S_{X}$, как $\theta(g)(x) = gx$ (тут нужны формальные проверки). Так мы получаем равносильное определение действия группы на множестве.  

	\begin{definition} 
		Орбитой элемента $x \in X$ под действием $G$ называется множество $\Orb(x) = \{ gx \ \vert \  g \in G \}$. Количество элементов в данной орбите называется \emph{длиной орбиты} (в разных орбитах может быть разное количество элементов).
	\end{definition}

	\begin{lemma} 
		Любые две орбиты либо не пересекаются, либо совпадают. Таким образом, множество $X$разбивается в дизъюнктное объединение орбит.
	\end{lemma}

	\begin{definition} 
		 Неподвижными точками элемента $g \in G$ называются те $x \in X$, для которых $gx = x$. Множество неподвижных точек элемента $g$ мы будем обозначать через $\Fix(g)$.
	\end{definition}

	\begin{definition} 
	 	Множество элементов группы $G$, оставляющих на месте данный элемент $x \in X$ называется стабилизатором элемента $x$ и обозначается через $\Stab(x)$. Другими словами, $\Stab(x) = \{ g \in G \ \vert \ gx = x \}$. Очевидно, что стабилизатор является подгруппой в $G$.
	 \end{definition} 



	

	



	

	






	

	




