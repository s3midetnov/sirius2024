\subsection{Группы}

	И вот мы наконец плавно подошли к одному из главных определений в нашем курсе. 

	\begin{definition} 
		Множество $G$ с заданной на нём бинарной операцией $\circ$ (мы часто будем называть её умножением) называется \emph{группой}, если 
		\begin{enumerate}
			\item $(a \circ b) \circ c = a \circ (b \circ c) = a \circ b \circ c$. 
			\item Существует \emph{нейтральный} элемент $e \in G$, то есть такой элемент, что $e \circ g = g \circ e = g$ для всех элементов $g \in G$. 
			\item У всех элементов $g \in G$ есть обратный элемент $g^{-1}$, то есть такой, что $g \circ g^{-1} = g^{-1} \circ g = e$.
		\end{enumerate}
	\end{definition}

	\begin{remark}
		Заметим, что любая группа преобразований является группой. Группы бывают разные, но в дальнейшем, группы преобразований будут основным примером групп для нас. 
	\end{remark}

	\begin{example}
		Вновь глупые числовые примеры. Пояснение примеров из прошлого параграфа. 
	\end{example}

	\begin{remark}
		Вот в этом месте я бы хотел сделать такой разгон про то, зачем это всё на самом деле (зачем переходить от изучения \emph{конкретных преобразований} к \emph{групп, как обще алгебраических структур}. Хороший пример такого разгона в первых пяти минутах \href{https://www.youtube.com/watch?v=NKZ_2EzHaSI}{этой лекции}, гляньте пж. Мне кажется, что это \bf{очень важно} при изложении групп детям. 
	\end{remark} 

	После того, как мы вдоволь насладились примерами, можно начать изучать наши объекты с общей точки зрения. 

	\begin{observation} 
		Нейтральный элемент группы единственен. Обратный элемент к элементу группы $g \in G$ единственен. 
	\end{observation}

	\begin{remark}
		Какими аксиомами группы мы пользовались для доказательства? 
	\end{remark}

	\begin{definition} 
		Два элемента группы $a, b \in G$ называются \emph{коммутирующими} (или, перестановочными), если $ab = ba$. Если все элементы группы $G$ коммутируют между собой, $G$ называтеся \emph{Абелевой} или \emph{коммутативной}. 
	\end{definition}

	\begin{remark}
		Заметим, что \bf{не абелевы} группы бывают. Например, рассмотренная нами ранее группа симметрий треугольника $S_{3} \cong D_{3}$.
	\end{remark}

	\begin{example}
		Заметим, что $(ab)^{1} = b^{-1}a^{-1}$ (\emph{Сначала мы надеваем носок, а потом ботинок. С другой  стороны, сначала мы снимаем ботинок, а потом носок}).
	\end{example}

	\begin{observation}
		В группе можно сокращать равенство справа и слева. 
	\end{observation}


	\subsection{Группа перестановок}

	Мы об этом уже не раз говорили, теперь 

	\begin{definition} 
		Пусть $X$~--- множество. Множество взаимно однозначных отображений $X \to X$ образует группу (отностельно композиции), её мы будем называть \emph{симметрической группой} на множестве $X$ и обозначать $S_{X}$. 
	\end{definition}

	\begin{remark}
		Если множество $X$ конечно и в нём $n$ элементов, то мы можем думать про него, как про множество $\{ 1, 2, \ldots, n \}$.  Тогда перестановки можно записывать в виде табличек вида 
		\[
			\sigma = \begin{pmatrix} 1 & 2 & \ldots & n \\ i_{1} & i_{2} & \ldots & i_{n} \end{pmatrix}.
		\]

		Симметрическу группу множества $\{ 1, 2, \ldots, n \}$ мы будем обозначать, как $S_{n}$.
	\end{remark}

	\textcolor{red}{\bf{Про группу перестановок вообще много чего можно говорить. Чего мы хотим про неё говорить?}}

	\subsection{Циклическая группа}

	to be upd\ldots






	

	




