\subsection{Дальнейший план того, что я не написал еще}

	\begin{enumerate}
		\item Группы перестановок, группы движений, аффинные преобразования, строковые примеры.  
		\item Циклические группы, остатки, порядок группы. 
		\item Гомоморфизмы и изоморфизмы, как сохранение структуры (и как разный способ записывать одни и те же преобразования) . 
		\item Подгруппы.
		\item Действие группы на множестве, сопряжения, нормальные подгруппы. 

		\item (?) классы смежности, индекс, теорема Лагранжа. 
		\item (?) нормальные подгруппы и фактор. 
		\item (?) Теорема о гомоморфизме. 
		\item (?) Образующие и соотношения. 
	\end{enumerate}

	\noindent\bf{Лекции:}
	\begin{enumerate}
		\item Введение самая база преобразования $\Rightarrow$ группы моноиды 
		\item Примеры 1 
		\item Примеры 2 (где-то среди них разговор как задать группу и разговор про циклические группы)
		\item Примеры 3 + симметрические группы
		\item Гомоморфизмы + подгруппы
		\item Действие группы на множестве. Примеры. Орбиты, стабилизаторы, фиксаторы. Действие, как гомоморфизм в группу перестановок. 
		\item Всё еще действия. Больше примеров и какие-то содержательные факты про орбиты/стабилизаторы. 
		\item Свободная группа $\leftrightarrow$ языки, регулярные корневые деревья, самоподобие. 
		\item Автоматы (как преобразование языка или как модель машины с кончной памятью). Побольше примеров. 
		\item Рост,  %(\emph{осторожно модерн}). 

	\end{enumerate}

	Листики: 
	\begin{enumerate}
		\item Банальщина про биекции и группы и моноиды (такая-то операция дает что).
		\item Задачки про группы и перестановки.
		\item Пропаганда гомоморфизмов (всякие свойства и какие-то примеры).
		\item 

	\end{enumerate}
	Идеи для листочков (это артему добавить) действие группы на автомате дает симметрии языка + язык григорчука