	\bf{Общая конва такова:} следуем~\cite{alekseev}, добавляя нужные конкретно нам примеры. Делаем мощный упор в действие 
	
	\subsection{Преобразования}

	\begin{definition} 
		Пусть $M$~--- некоторое множество элементов произвольной природы. Если каждой упорядоченной паре элементов из $M$ поставлен в соответствие определённый элемент также из $M$, то говорят, что на $M$ задана бинарная операция (обозначим её за $\circ)$.

		Говоря чуть более взрослым языком, бинарная операция~--- это отображение 
		\[
			\circ\colon M \times M \to M, \quad (m, m) \mapsto m \circ m \in M.
		\]
	\end{definition}

	\begin{example}
		Для начала, всякие скучные числовые примеры. 
	\end{example}

	Причём же тут преобразования? Оказывается, бинарные операции~--- удобный способ записывать, что происходит, когда мы проделываем последовательно много различных преобразований.  Рассмотрим сначала некоторые примеры. 

	\begin{example}
		Повороты равностороннего треугольника, переводящие его в себя, \emph{переставляют} вершины. Композиция преобразований~--- бинарная операция. Какие у неё свойства? Выпишем \emph{таблицу умножения} (см.~\cite[пример 1]{alekseev}).

		А еще есть симметрии, их тоже можно композицировать (и композицировать с поворотами).  
	\end{example}

	\textcolor{red}{Коллеги, сюда напишите Ваш любимый пример про строки и любимый пример про графы (по штуке, соовтественно).}

	Теперь давайте подумаем, какие свойства ествественно было бы требовать от преобразований.

	\begin{itemize}
		\item Если мы говорим о преобразованиях одного типа, то естественно требовать, чтобы когда мы проделывали несколько одно за другим, получалось преобразование того же типа (как композиция поворотов~--- поворот). 

		\item Во всех примерах мы видели, что есть \emph{тождественное} преобразование (которое просто не делает ничего). 

		\item Кроме того, композиция преобразований естественным образом \emph{ассоциативна}. То есть, для любых преобразований $f, g, h$ мы имеем $f \circ (g \circ h) = (f \circ g) \circ h = f \circ g \circ h$. 

		\item В примерах мы также видели, что для каждого преобразования $g$ существует преобразование $h$, которое после применения $g$ возвращает ситуацию в исходный вид. 
	\end{itemize}

	Теперь попробуем сделать наши примеры немного более строгими. 

	\begin{definition} 
		Напоминиание про функции, инъекции, сюръекции, биекции. 
	\end{definition}

	Приведём какой-нибудь не слишком скучный пример. 

	\begin{example}
		Пусть отображение $\varphi$ ставит в соответствие каждому городу мира первую букву из его названия на русском языке (например, $\varphi$(Санкт-Петербург) = С). Будет ли $\varphi$ отображением всех городов мира \bf{на} весь русский алфавит?

		Нет, не будет (так как едва ли есть город, начинающийся на ъ, например). Будет ли это отображение инъективным? Очевидно, что тоже нет. 
	\end{example}

	Пока что мы понимали слово \emph{преобразование} наивно, дадим теперь строгое математическое определение. 

	\begin{definition} 
		Произвольное взаимно однозначное отображение множества $M$ на себя, $g\colon M \to M$ , мы будем для краткости называть \emph{преобразованием} множества $M$.
	\end{definition}

	\begin{example}
		Если множество $M$ конечное, то можно писать табличку и будет перестановка (слово перестановка тут еще не говорим), но тем не менее. 
	\end{example}

	\begin{definition} 
		Так как преобразование~--- это взаимно однозначное отображение, то для каждого преобразования $g$ существует обратное преобразование $g^{-1}$, которое определяется следующим образом: если $g(A) = B$, то $g^{-1}(B) = A$.
	\end{definition}

	\begin{example}
		Выписать обратное преобразование для какой-нибудь композиции поворота и симметрии. 
	\end{example}

	Если у нас есть некоторое фиксированное множество $M$ и все его преобразования, то мы можем определить их произведение (композицию): 
	\[
		(g_1 g_2)(A) = g_1(g_2(A)),
	\]
	то есть сначала делаем $g_2$, а потом $g_1$. 

	\begin{definition} 
		Пусть некоторое множество преобразований $G$ таково, что 
		\begin{enumerate}
		 	\item если преобразования $g_1$ и $g_2$ содержатся в $G$, то и их произведение $g_3 = g_1g_2$ содержится в $G$;
		 	\item если преобразование $g$ содержится в $G$, то и обратное ему преобразование $g^{-1}$ содержится в $G$.
		 \end{enumerate}
		 Тогда такое множество преобразований $G$ мы будем называть \emph{группой преобразований}.
	\end{definition}
	

	