	\bf{Общая конва такова:} следуем~\cite{alekseev}, добавляя нужные конкретно нам примеры. Делаем мощный упор в действие 
	
	\subsection{Преобразования}

	\begin{definition} 
		Пусть $M$~--- некоторое множество элементов произвольной природы. Если каждой упорядоченной паре элементов из $M$ поставлен в соответствие определённый элемент также из $M$, то говорят, что на $M$ задана бинарная операция (обозначим её за $\circ)$.

		Говоря чуть более взрослым языком, бинарная операция~--- это отображение 
		\[
			\circ\colon M \times M \to M, \quad (m, m) \mapsto m \circ m \in M.
		\]
	\end{definition}

	\begin{example}
		Для начала, всякие скучные числовые примеры. 
	\end{example}

	Причём же тут преобразования? Оказывается, бинарные операции~--- удобный способ записывать, что происходит, когда мы проделываем последовательно много различных преобразований.  Рассмотрим сначала некоторые примеры. 

	\begin{example}
		Повороты равностороннего треугольника, переводящие его в себя, \emph{переставляют} вершины. Композиция преобразований~--- бинарная операция. Какие у неё свойства? Выпишем \emph{таблицу умножения} (см.~\cite[пример 1]{alekseev}).

		А еще есть симметрии, их тоже можно композицировать (и композицировать с поворотами).  
	\end{example}

	\textcolor{red}{Коллеги, сюда напишите Ваш любимый пример про строки и любимый пример про графы (по штуке, соовтественно).}

	Теперь давайте подумаем, какие свойства ествественно было бы требовать от преобразований.

	\begin{itemize}
		\item Если мы говорим о преобразованиях одного типа, то естественно требовать, чтобы когда мы проделывали несколько одно за другим, получалось преобразование того же типа (как композиция поворотов~--- поворот). 

		\item Во всех примерах мы видели, что есть \emph{тождественное} преобразование (которое просто не делает ничего). 

		\item Кроме того, композиция преобразований естественным образом \emph{ассоциативна}. То есть, для любых преобразований $f, g, h$ мы имеем $f \circ (g \circ h) = (f \circ g) \circ h = f \circ g \circ h$. 

		\item В примерах мы также видели, что для каждого преобразования $g$ существует преобразование $h$, которое после применения $g$ возвращает ситуацию в исходный вид. 
	\end{itemize}

	Теперь попробуем сделать наши примеры немного более строгими. 

	\begin{definition} 
		Напоминиание про функции, инъекции, сюръекции, биекции. 
	\end{definition}

	Приведём какой-нибудь не слишком скучный пример. 

	\begin{example}
		Пусть отображение $\varphi$ ставит в соответствие каждому городу мира первую букву из его названия на русском языке (например, $\varphi$(Санкт-Петербург) = С). Будет ли $\varphi$ отображением всех городов мира \bf{на} весь русский алфавит?

		Нет, не будет (так как едва ли есть город, начинающийся на ъ, например). Будет ли это отображение инъективным? Очевидно, что тоже нет. 
	\end{example}

	Пока что мы понимали слово \emph{преобразование} наивно, дадим теперь строгое математическое определение. 

	\begin{definition} 
		Произвольное взаимно однозначное отображение множества $M$ на себя, $g\colon M \to M$ , мы будем для краткости называть \emph{преобразованием} множества $M$.
	\end{definition}

	\begin{example}
		Если множество $M$ конечное, то можно писать табличку и будет перестановка (слово перестановка тут еще не говорим), но тем не менее. 
	\end{example}

	\begin{definition} 
		Так как преобразование~--- это взаимно однозначное отображение, то для каждого преобразования $g$ существует обратное преобразование $g^{-1}$, которое определяется следующим образом: если $g(A) = B$, то $g^{-1}(B) = A$.
	\end{definition}

	\begin{example}
		Выписать обратное преобразование для какой-нибудь композиции поворота и симметрии. 
	\end{example}

	Если у нас есть некоторое фиксированное множество $M$ и все его преобразования, то мы можем определить их произведение (композицию): 
	\[
		(g_1 g_2)(A) = g_1(g_2(A)),
	\]
	то есть сначала делаем $g_2$, а потом $g_1$. 

	\begin{definition} 
		Пусть некоторое множество преобразований $G$ таково, что 
		\begin{enumerate}
		 	\item если преобразования $g_1$ и $g_2$ содержатся в $G$, то и их произведение $g_3 = g_1g_2$ содержится в $G$;
		 	\item если преобразование $g$ содержится в $G$, то и обратное ему преобразование $g^{-1}$ содержится в $G$.
		 \end{enumerate}
		 Тогда такое множество преобразований $G$ мы будем называть \emph{группой преобразований}.
	\end{definition}

	\subsection{Группы. Определение, примеры, базовые конструкции.}

	И вот мы наконец плавно подошли к одному из главных определений в нашем курсе. 

	\begin{definition} 
		Множество $G$ с заданной на нём бинарной операцией $\circ$ (мы часто будем называть её умножением) называется \emph{группой}, если 
		\begin{enumerate}
			\item $(a \circ b) \circ c = a \circ (b \circ c) = a \circ b \circ c$. 
			\item Существует \emph{нейтральный} элемент $e \in G$, то есть такой элемент, что $e \circ g = g \circ e = g$ для всех элементов $g \in G$. 
			\item У всех элементов $g \in G$ есть обратный элемент $g^{-1}$, то есть такой, что $g \circ g^{-1} = g^{-1} \circ g = e$.
		\end{enumerate}
	\end{definition}

	\begin{remark}
		Заметим, что любая группа преобразований является группой. Группы бывают разные, но в дальнейшем, группы преобразований будут основным примером групп для нас. 
	\end{remark}

	\begin{example}
		Вновь глупые числовые примеры ($\Z, \Q, \R$). Пояснение примеров из прошлого параграфа. 
	\end{example}

	\begin{remark}
		Вот в этом месте я бы хотел сделать такой \bf{разгон} про то, зачем это всё на самом деле (зачем переходить от изучения \emph{конкретных преобразований} к изучению \emph{групп, как обще алгебраических структур}. Хороший пример такого разгона в первых пяти минутах \href{https://www.youtube.com/watch?v=NKZ_2EzHaSI}{этой лекции}, гляньте пж. Мне кажется, что это \bf{очень важно} при изложении групп детям. 
	\end{remark} 

	После того, как мы вдоволь насладились примерами, можно начать изучать наши объекты с общей точки зрения. 

	\begin{observation} 
		Нейтральный элемент группы единственен. Обратный элемент к элементу группы $g \in G$ единственен. 
	\end{observation}

	\begin{remark}
		Какими аксиомами группы мы пользовались для доказательства? 
	\end{remark}

	\begin{definition} 
		Два элемента группы $a, b \in G$ называются \emph{коммутирующими} (или, перестановочными), если $ab = ba$. Если все элементы группы $G$ коммутируют между собой, $G$ называтеся \emph{Абелевой} или \emph{коммутативной}. 
	\end{definition}

	\begin{remark}
		Заметим, что \bf{не абелевы} группы бывают. Например, рассмотренная нами ранее группа симметрий треугольника $S_{3} \cong D_{3}$.
	\end{remark}

	\begin{example}
		Заметим, что $(ab)^{1} = b^{-1}a^{-1}$ (\emph{Сначала мы надеваем носок, а потом ботинок. С другой  стороны, сначала мы снимаем ботинок, а потом носок}).

		В качестве проверки понимания можно задать слушателям такой вопрос:  (\emph{очень уж много носков}) Пусть $a_1, a_2, \ldots, a_n$~--- элементы некоторой группы $A$. Какой элемент будет обратным к $a_1 a_2 \ldots a_n$?
	\end{example}

	\begin{observation}
		В группе можно сокращать равенство справа и слева. 
	\end{observation}


	\begin{example}
		\begin{definition} 
		Пусть $X$~--- множество. Множество взаимно однозначных отображений $X \to X$ образует группу (отностельно композиции), её мы будем называть \emph{симметрической группой} на множестве $X$ и обозначать $S_{X}$. 
	\end{definition}

	\begin{remark}
		Если множество $X$ конечно и в нём $n$ элементов, то мы можем думать про него, как про множество $\{ 1, 2, \ldots, n \}$.  Тогда перестановки можно записывать в виде табличек вида 
		\[
			\sigma = \begin{pmatrix} 1 & 2 & \ldots & n \\ i_{1} & i_{2} & \ldots & i_{n} \end{pmatrix}.
		\]

		Симметрическу группу множества $\{ 1, 2, \ldots, n \}$ мы будем обозначать, как $S_{n}$.
	\end{remark}
	\end{example}

	Теперь рассмотрим еще одну важную конструкцию. 

	\begin{definition} 
		Пусть $(G_1, \circ), (G_2, *)$~--- группы. Тогда их \emph{прямое произведение}~--- это группа, как множество совпадающая с $G_1 \times G_2$, опреации на котором вводятся покомпонентно.   
	\end{definition}

	\begin{example}
		$\Z \times \Z$, например. 
	\end{example}
	

	\subsection{Порождение, циклические группы, порядок элемента}

	\begin{definition} 
		 Пусть $S$~--- подмножество группы $G$. Подгруппой, порожденной множеством $S$, называется наименьшая подгруппа в $G$, содержащая $S$.  Её мы будем обозначать, как $\langle S \rangle$.
	\end{definition}

	\begin{definition} 
		Пусть $G$~--- группа, $S$~--- подмножество. Подгруппа, порожденная $S$~--- это наименьшая подгруппа в $G$. Мы будем обозначать её, как $\langle S \rangle$. 
	\end{definition}

	\begin{statement} 
		$\langle S \rangle$ состоит из всех элементов $s_1 s_2 \ldots s_k$, где $k \in \N$, а $s _i \in S \cup S^{-1}$. 
	\end{statement}

	\begin{remark}
		Переписать это в абелеву нотацию. 
	\end{remark}

	\begin{definition} 
		Порядком элемента $g$ группы $G$ называют наименьшее натуральное $n$, для которого $g^n = e$. Если такого $n$ не существует, то говорят, что $g$ имеет бесконечный порядок. 

		Порядок элемента $g$ мы будем обозначать, как $\ord{g}$.
	\end{definition}

	\begin{example}
		Например, порядок элемнта $2$ в группе $\Z/10$ равен 5. А вот в группе $\Z$ любой ненулевой элемент имеет бесконечный порядок. 
	\end{example} 

	\begin{example}
		Рассмотрим правильный $n$-угольник и все повороты плоскости, переводящие его в себя. Заметим, что если мы возьмём поворот $R = R_{\frac{2\pi}{n}}$, то 

		\[
			R^n = R_{2\pi} = e,\  R^{n + 1} = R, \ R^{n + 2} = R^2 
		\]
		и так мы получим в точности все повороты, оставляющие $n$-угольник на месте. 
	\end{example}

	\begin{definition} 
		Группу $G$, состоящую из элементов $e, a^1, \ldots, a^{n - 1}$ (где элемент $a$ имеет порядок $n$) называют \emph{циклической группой порядка $n$, порожденной элементом $a$} (а обозначать её мы будем, как $C_n$). Элемент $a$ называется \emph{образующей} этой группы. 
	\end{definition}

	Как видно из примера выше, повороты правильного $n$-угольника образуют циклическую группу порядка $n$.

	\begin{remark}
		А еще бывает \emph{бесконечная циклическая группа}, это $\Z$ (а образующая у неё $1$). Про неё логично думать именно так, так как любой её элемент можно представить в виде $n \cdot 1$, где $n$~--- целое число.
	\end{remark}

	Чуть позже иы увидим, что подгруппа порожденная одним элементом $g$~--- это в точности циклическая группа порядка $\ord{g}$. 

	\subsection{Гомоморфизм, изоморфизм}

	Мы уже видели, что между многими группами можно построить теоретико-множественную биекцию, но ведь это не говорит нам, что эти группы одинаковые. 

	\begin{example}
		Ну, например есть группа $\Z/6$ и группа $S_{3}$. Так как обе группы имеют порядок 6, между ними легко построить теоретико-множественную биекцию, но видно, что группы на самом деле существенно разные (можно привести какой-то пример, где элементы разного порядка, или что-то в таком духе). 
	\end{example}

	\begin{definition} 
		Отображение $f$ между группами $(G, \circ)$ и $(H, *)$ называется \emph{гомоморфизмом}, если 
		\[
			\forall a, b \in G \quad  f(a \circ b) = f(a) * f(b).
		\]
	\end{definition}

	Условие в определении гомоморфизм говорит о \emph{сохранении стркутуры}. 

	\begin{definition} 
		Пусть $\varphi\colon G \to H$~--- гомоморфизм групп. Его \emph{ядром} называют $\varphi^{-1}(e_{H}) \subset G$. Иными словами, 
		\[
			\Ker{\varphi} \eqdef \{ g \in G \ \vert\  \varphi(g) = e_{H} \} = \varphi^{-1}(e_{H}).
		\]
		\emph{Образом} $\varphi$ называют его образ, как функции, то есть 
		\[
			\mathrm{Im}{\varphi} = \{ \varphi(g) \ \vert\  g \in G \} \subset G
		\]
	\end{definition}
	

	