\documentclass[12pt]{article}

\input{print/preamble}
\begin{document}

\centerline{\bf{Серия 1. Вводная и предпраздничная. }}

	\epigraph{Symmetry is a vast subject, significant in art and nature. Mathematics lies at its root, and it would be hard to find a better one on which to demonstrate the working of the mathematical intellect.}{Герман Вейль}

	\bf{0.} Докажите, что композиция отображений ассоциативна, т.е., что для $f \colon X \to Y, g \colon Y \to Z, \ h \colon Z \to T$
	\[
		h \circ (g \circ f) = (h \circ g) \circ f.
	\]

	\bf{1.} а) Докажите, что нейтральный элемент группы единственнен. б) Докажите, что для любого элемента $g \in G$ существует единственный обратный элемент $g^{-1}$. 

	\bf{2.} а) Опишите (словами и геометрически) группу симметрий квадрата. б) Найдите в ней такой элемент $x$, что $x^3 = R_{90^{\circ}}$. в) Найдите три симметрии квадрата $f, g, h \in D_{4}$, для которых $fg = gh$, но $f \neq h$.

	\begin{definition} 
		\emph{Полугруппой} называют множество с ассоциативной бинарной операцией. 
	\end{definition}

	\bf{3.} а)  Пусть в группе $G$ для любого $g \in G$ выполнено $g^2 = e$. Докажите, что $G$~--- абелева группа.  б) Пусть $G$~--- конечная полугруппа. Докажите, что существует такой элемент $g \in G$, что $g^2 = e$.

	\bf{4.} Вася нарисовал бесконечно длинную полоску из букв $H$ на одинаковом расстоянии.
	\[
		\text{\ldots\  H \ H \ H \ H \ H \ H \ H \ H \ H \ldots}
	\]
	Помогите Васе описать группу симметрий этого рисунка. 



	




\end{document}