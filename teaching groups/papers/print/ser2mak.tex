\documentclass[10pt]{article}

\input{preamble}
\usepackage{multicol}
\usepackage{pdflscape}
\usepackage{wrapfig}
\begin{document}
	\begin{landscape}

	\pagestyle{empty}
	\begin{multicols}{2}

	\centerline{\bf{Серия 2. Нормальное завтра. }}

	\epigraph{Whenever groups disclose themselves, or could be introduced, simplicity crystallized out of comparative chaos.}{E. T. Bell}

	%\bf{0.} Пусть $\varphi \colon G \to H$~--- гомоморфизм групп. а) Докажите, что $\Ker{\varphi}$~--- подгруппа в $G$. б) Докажите, что $\Im{\varphi}$~--- подгруппа в $H$. в) Докажите, что $\varphi$ инъективен тогда и только тогда, когда $\Ker{\varphi} = \{ e_{G} \}$. 

	\bf{0.} а) Верно ли, что $C_{2} \times C_{4} \cong C_{8}$? б) Опишите условия, при которых  $\Z/m \times \Z/n \cong \Z/nm$ (и докажите изоморфность в этом случае). 


	\bf{1.} Пусть $G$~--- конечная абелева группа, в которой в точности один элемент $f$ порядка 2. Докажите, что 
	\[
		\prod_{g \in G} g = f.
	\]

	\bf{2.} Пусть $G$~--- группа чётного порядка. Докажите, что в ней есть элемент порядка $2$. 

	\bf{3.} Пусть $g \in G$~--- элемент нечётного порядка. Что можно сказать о порядке $g^2$?

	 \bf{4.}  Пусть $N$~--- подгруппа группы $G$, будем говорить, что она \emph{удовлетворяет свойству $(\cH)$}, если $\forall g \in G \ \forall n \in N$ существует такой $n' \in N$, что $gn = n'g$.

	Для подгруппы $H$ и элемента $g$ будем обозначать $g H = \{ gh \ \vert \ h \in H \}$, а для $R \subset G$ будем обозначать $RH = \{ r h \vert \ r \in R, h \in H \}$.

	а) Докажите, что если $N$ это подгруппа, то
	\vspace{-2mm}
	\begin{enumerate}
		\item $NN = N$. 
		\item $N^{-1} = N$. 
		\item Если $\forall g\in G\  \forall n \in N\colon g n g^{-1} \in N$, то $N$ удовлетворяет свойству $(\cH)$.
	\end{enumerate}

	б) Докажите, что если $N$ удовлетворяет свойству $(\cH)$, то
	\vspace{-2mm}
	\begin{enumerate}
		\item $\forall g \ gN = Ng$.
		\item $(gN)(hN) = (gh)N$ и приведите пример, когда это не так если $N$ не удовлетворяет условию $(\cH)$.
	\end{enumerate}

	в) Пусть $\varphi \colon G \to H$~--- гомоморфизм групп, $N = \Ker{\varphi}$. Докажите, что $N$ удовлетворяет свойству $(\cH)$.

	\bf{5.} Неряшливый преподаватель выписал на доску список из девяти целых чисел, образующих группу по умножению по модулю $91$. К сожалению, он забыл выписать одно из чисел и на доске были выписаны лишь числа $1, 9, 16, 22, 53, 74, 79, 81$. Какое число он забыл написать? 

	\begin{definition} 
		\emph{Моноидом} называется множество $M$ с ассоциативной бинарной операцией и нейтральным элементом.   
	\end{definition}

	\bf{6.} Пусть $M_1, M_2$~--- моноиды. Отображение $\varphi \colon M_1 \to M_2$ назовём \emph{хорошим}, если $\forall a, b \in M_1 \ \varphi(ab) = \varphi(a) \varphi(b)$
	  Верно ли, что если $\varphi$~--- хорошее отображение, то $\varphi(e_{M_1}) = e_{M_2}$?

	 \bf{7.}  Пусть $G, H$~--- группы,  $G \cong H \times G$. Можно ли из этого заключиь, что $H$~--- тривиальная группа. (\emph{Подсказска.} Нет! Попробуйте построить контрпример.)

	 \bf{8.} Докажите, что $(\Q, +)$ не может быть представлена в виде произведения двух нетривиальных групп. 


	\columnbreak

	
	\centerline{\bf{Серия 2. Нормальное завтра. }}

	\epigraph{Whenever groups disclose themselves, or could be introduced, simplicity crystallized out of comparative chaos.}{E. T. Bell}

	%\bf{0.} Пусть $\varphi \colon G \to H$~--- гомоморфизм групп. а) Докажите, что $\Ker{\varphi}$~--- подгруппа в $G$. б) Докажите, что $\Im{\varphi}$~--- подгруппа в $H$. в) Докажите, что $\varphi$ инъективен тогда и только тогда, когда $\Ker{\varphi} = \{ e_{G} \}$. 

	\bf{0.} а) Верно ли, что $C_{2} \times C_{4} \cong C_{8}$? б) Опишите условия, при которых  $\Z/m \times \Z/n \cong \Z/nm$ (и докажите изоморфность в этом случае). 


	\bf{1.} Пусть $G$~--- конечная группа, в которой в точности один элемент $f$ порядка 2. Докажите, что 
	\[
		\prod_{g \in G} g = f.
	\]

	\bf{2.} Пусть $G$~--- группа чётного порядка. Докажите, что в ней есть элемент порядка $2$. 

	\bf{3.} Пусть $g \in G$~--- элемент нечётного порядка. Что можно сказать о порядке $g^2$?

	 \bf{4.}  Пусть $N$~--- подгруппа группы $G$, будем говорить, что она \emph{удовлетворяет свойству $(\cH)$}, если $\forall g \in G \ \forall n \in N$ существует такой $n' \in N$, что $gn = n'g$.

	Для подгруппы $H$ и элемента $g$ будем обозначать $g H = \{ gh \ \vert \ h \in H \}$, а для $R \subset G$ будем обозначать $RH = \{ r h \vert \ r \in R, h \in H \}$.

	а) Докажите, что если $N$ это подгруппа, то
	\vspace{-2mm}
	\begin{enumerate}
		\item $NN = N$. 
		\item $N^{-1} = N$. 
		\item Если $\forall g\in G\  \forall n \in N\colon g n g^{-1} \in N$, то $N$ удовлетворяет свойству $(\cH)$.
	\end{enumerate}

	б) Докажите, что если $N$ удовлетворяет свойству $(\cH)$, то
	\vspace{-2mm}
	\begin{enumerate}
		\item $\forall g \ gN = Ng$.
		\item $(gN)(hN) = (gh)N$ и приведите пример, когда это не так если $N$ не удовлетворяет условию $(\cH)$. 
	\end{enumerate}

	в) Пусть $\varphi \colon G \to H$~--- гомоморфизм групп, $N = \Ker{\varphi}$. Докажите, что $N$ удовлетворяет свойству $(\cH)$.

	\bf{5.} Неряшливый преподаватель выписал на доску список из девяти целых чисел, образующих группу по умножению по модулю $91$. К сожалению, он забыл выписать одно из чисел и на доске были выписаны лишь числа $1, 9, 16, 22, 53, 74, 79, 81$. Какое число он забыл написать? 

	\begin{definition} 
		\emph{Моноидом} называется множество $M$ с ассоциативной бинарной операцией и нейтральным элементом.   
	\end{definition}

	\bf{6.} Пусть $M_1, M_2$~--- моноиды. Отображение $\varphi \colon M_1 \to M_2$ назовём \emph{хорошим}, если $\forall a, b \in M_1 \ \varphi(ab) = \varphi(a) \varphi(b)$
	  Верно ли, что если $\varphi$~--- хорошее отображение, то $\varphi(e_{M_1}) = e_{M_2}$?

	 \bf{7.}  Пусть $G, H$~--- группы,  $G \cong H \times G$. Можно ли из этого заключиь, что $H$~--- тривиальная группа. (\emph{Подсказска.} Нет! Попробуйте построить контрпример.)

	 \bf{8.} Докажите, что $(\Q, +)$ не может быть представлена в виде произведения двух нетривиальных групп. 
	

	\end{multicols}
	\end{landscape}


\end{document}