\documentclass[10pt]{article}



%вставка изображений и картинок всяких

\usepackage{graphicx}
\newcommand{\incfig}[1]{%
    \def\svgscale{1.5}
    \import{./figures/}{#1.pdf_tex}
}
\graphicspath{{pictures/}}
\DeclareGraphicsExtensions{.pdf,.png,.jpg, .jpeg, .tex}

\usepackage{booktabs} % для таблиц
\usepackage{enumitem} % для списков


% Шрифты

\usepackage[english,russian]{babel}
\usepackage[T1]{fontenc}
\usepackage{libertine}


\usepackage{titling} % для \maketitle
\usepackage{textcomp}% специальные символы в тексте

\usepackage{mathtext}
\usepackage{amsmath,amsfonts,amssymb,amsthm,mathtools} % математика
\usepackage{icomma} % умная запятая
\usepackage{import} %  импортирование


\usepackage{pdfpages} % мультри-пдф страницы
\usepackage{transparent} % что-то про цвета


\usepackage{caption} % комментарии к figure
\usepackage{epigraph} % эпиграфы

\usepackage{comment} % удобные комментарии
\usepackage{xfrac} % дроби
\usepackage{moresize} % все размеры шрифтов
\usepackage{dsfont} % mathbb для всего


% Окружения для математики:

\newtheorem{statement}{Предложение}
\newtheorem{corollary}{Следствие}
\newtheorem{theorem}{Теорема}
\theoremstyle{definition}
\newtheorem{definition}{Определение}


\newtheorem{example}{Пример}
\newtheorem{homework}{Домашнее задание}
\newtheorem{antiexample}{Антиример}
\newtheorem{lemma}{Лемма}
\theoremstyle{remark}
\newtheorem*{remark}{Замечание}
\newtheorem*{exercise}{Упражнение}

% Настройки счетчиков:

%\numberwithin{equation}{section} % Number equations within sections (i.e. 1.1, 1.2, 2.1, 2.2 instead of 1, 2, 3, 4)
%\numberwithin{figure}{section} % Number figures within sections (i.e. 1.1, 1.2, 2.1, 2.2 instead of 1, 2, 3, 4)
%\numberwithin{table}{section} % Number tables within sections (i.e. 1.1, 1.2, 2.1, 2.2 instead of 1, 2, 3, 4)

% Геометрия файла:

\usepackage{geometry}

\setlist{noitemsep} % No spacing between list items

\geometry{left=1.5cm,right=1.5cm,top=2.5cm,bottom=2cm, a4paper}

% Счётчики разделов:


%\sectionfont{\vspace{6pt}\centering\normalfont\scshape} % \section{} styling
%\subsectionfont{\normalfont\bfseries} % \subsection{} styling
%\subsubsectionfont{\normalfont\itshape} % \subsubsection{} styling
%\paragraphfont{\normalfont\scshape} % \paragraph{} styling

\newcommand{\RNumb}[1]{\uppercase\expandafter{\romannumeral #1\relax}}

\renewcommand\thesection{\arabic{section}.}
\renewcommand\thesubsection{\thesection\arabic{subsection}}
\renewcommand\thesubsubsection{\RNumb{\arabic{subsubsection}}}
\renewcommand{\bf}{\textbf}

% Колонтитулы

\usepackage{fancyhdr}
\pagestyle{fancy}
\fancyhf{} % clear all fields
\fancyhead[L]{\rightmark}
\fancyhead[R]{\thepage}

\renewcommand{\sectionmark}[1]{%
  \markright{\thesection\ #1}}%
\setlength{\headheight}{17.0pt}
\addtolength{\topmargin}{-2.49998pt}



% Операторы:

\DeclareMathOperator{\ord}{ord}
\DeclareMathOperator{\ld}{ld}
\DeclareMathOperator{\id}{id}
\DeclareMathOperator{\exi}{exi}
\DeclareMathOperator{\osc}{osc}
\DeclareMathOperator{\num}{num}
\DeclareMathOperator{\Char}{char}
\DeclareMathOperator{\card}{Card}
\DeclareMathOperator{\sk}{sk}
\DeclareMathOperator{\den}{den}
\DeclareMathOperator{\essup}{essup}
\DeclareMathOperator{\ran}{ran}
\DeclareMathOperator{\rank}{rank}
\DeclareMathOperator{\dom}{dom}
\DeclareMathOperator{\diam}{diam}
\DeclareMathOperator{\dist}{dist}
\DeclareMathOperator{\disc}{disc}
\DeclareMathOperator{\rad}{rad}
\DeclareMathOperator{\supp}{supp}

\DeclareMathOperator{\sign}{sign}
\DeclareMathOperator{\Int}{Int}
\DeclareMathOperator{\RelInt}{RelInt}
\DeclareMathOperator{\Cl}{Cl}
\DeclareMathOperator{\Class}{\mathcal{C}\mathbf{\ell}}
\DeclareMathOperator{\CW}{CW}
\DeclareMathOperator{\Ideals}{Ideals}
\DeclareMathOperator{\pr}{pr}
\DeclareMathOperator{\ind}{ind}
\DeclareMathOperator{\Af}{Aff}
\DeclareMathOperator{\Aut}{Aut}
\renewcommand{\Im}{\mathop{\mathrm{Im}}\nolimits}
\DeclareMathOperator{\Conv}{conv}
\DeclareMathOperator{\Fr}{Fr}
\DeclareMathOperator{\Tr}{Tr}
\DeclareMathOperator{\Nm}{\mathrm{N}}
\DeclareMathOperator{\Span}{span}
\DeclareMathOperator{\Map}{Map}
\DeclareMathOperator{\Hom}{Hom}
\DeclareMathOperator{\Ker}{Ker}
\DeclareMathOperator{\Ext}{Ext}
\DeclareMathOperator{\Div}{Div}
\DeclareMathOperator{\NRad}{NRad}
\DeclareMathOperator{\Coker}{Coker}
\DeclareMathOperator{\Gal}{Gal}
\DeclareMathOperator{\Specm}{Specm}
\DeclareMathOperator{\Spec}{Spec}
\DeclareMathOperator{\Ht}{ht}
\DeclareMathOperator{\End}{End}
\DeclareMathOperator{\Rad}{Rad}
\DeclareMathOperator{\Ann}{Ann}
\DeclareMathOperator{\Vol}{Vol}
\DeclareMathOperator{\res}{res}
\DeclareMathOperator{\Gr}{Gr}
\DeclareMathOperator{\Bl}{Bl}
\DeclareMathOperator{\mult}{mult}
\DeclareMathOperator{\cont}{cont}
\DeclareMathOperator{\area}{area}

\renewcommand{\Re}{\mathop{\mathrm{Re}}\nolimits}
\DeclarePairedDelimiter\lr{(}{)}
\DeclareRobustCommand{\divby}{%
     \mathrel{\text{\vbox{\baselineskip.65ex\lineskiplimit0pt\hbox{.}\hbox{.}\hbox{.}}}}%
}
\newcommand{\eqdef}{\stackrel{\mathrm{def}}{=}}
\DeclareRobustCommand{\notdivby}{%
     \!\!\not\;\divby%
}
\newcommand{\lei}{\trianglelefteq}

%%%% гиперссылки
\usepackage{xcolor} % цвета
\usepackage[unicode, pdftex]{hyperref}
\hypersetup{%
  colorlinks=false,
  linkbordercolor=cyan
}

% Буковы

\newcommand{\N}{\mathbb{N}}			 		
\newcommand{\Z}{\mathbb{Z}}			
\newcommand{\Q}{\mathbb{Q}}		
\newcommand{\R}{\mathbb{R}}	
\let\oldC\C
\renewcommand{\C}{\mathbb{C}} 				
\newcommand{\F}{\mathbb{F}}	
 
\let\oldAA\AA
\renewcommand{\AA}{\mathbb{A}}				
\newcommand{\DD}{\mathbb{D}}  						
\newcommand{\EE}{\mathbb{E}} 			
\newcommand{\HH}{\mathbb{H}}					
\newcommand{\KK}{\mathbb{K}} 					
\newcommand{\OO}{\mathbb{O}} 		
\newcommand{\PP}{\mathbb{P}}			
\let\oldSS\SS		
\renewcommand{\SS}{\mathbb{S}}						
\newcommand{\TT}{\mathbb{T}} 			

\newcommand{\cA}{\mathcal{A}}
\newcommand{\cB}{\mathcal{B}}
\newcommand{\cC}{\mathcal{C}}
\newcommand{\cD}{\mathcal{D}}
\newcommand{\cE}{\mathcal{E}}
\newcommand{\cF}{\mathcal{F}}
\newcommand{\cG}{\mathcal{G}}
\newcommand{\cH}{\mathcal{H}}
\newcommand{\cI}{\mathcal{I}}
\newcommand{\cJ}{\mathcal{J}}
\newcommand{\cK}{\mathcal{K}}
\newcommand{\cL}{\mathcal{L}}
\newcommand{\cM}{\mathcal{M}}
\newcommand{\cN}{\mathcal{N}}
\newcommand{\cO}{\mathcal{O}}
\newcommand{\cQ}{\mathcal{Q}}
\newcommand{\cP}{\mathcal{P}}
\newcommand{\cR}{\mathcal{R}}
\newcommand{\cS}{\mathcal{S}}
\newcommand{\cT}{\mathcal{T}}
\newcommand{\cU}{\mathcal{U}}
\newcommand{\cV}{\mathcal{V}}
\newcommand{\cW}{\mathcal{W}}
\newcommand{\cX}{\mathcal{X}}
\newcommand{\cY}{\mathcal{Y}}
\newcommand{\cZ}{\mathcal{Z}}
			
\newcommand{\rD}{\mathrm{D}}
\newcommand{\rK}{\mathrm{K}}			
\newcommand{\rP}{\mathrm{P}}
\newcommand{\rT}{\mathrm{T}}			

\newcommand{\fA}{\mathfrak{A}}
\newcommand{\fQ}{\mathfrak{Q}}
\newcommand{\fB}{\mathfrak{B}}
\newcommand{\fT}{\mathfrak{T}}
\newcommand{\fK}{\mathfrak{K}}
\newcommand{\fM}{\mathfrak{M}}
\newcommand{\fL}{\mathfrak{L}}
\newcommand{\fR}{\mathfrak{R}}
\newcommand{\fP}{\mathfrak{P}}
\newcommand{\fC}{\mathfrak{C}}
\newcommand{\fX}{\mathfrak{X}}
\newcommand{\fS}{\mathfrak{S}}

\newcommand{\fm}{\mathfrak{m}}
\newcommand{\fb}{\mathfrak{b}}
\newcommand{\ff}{\mathfrak{f}}
\newcommand{\fp}{\mathfrak{p}}
\newcommand{\fq}{\mathfrak{q}}
\newcommand{\fh}{\mathfrak{h}}
\newcommand{\fo}{\mathfrak{o}}
\newcommand{\fe}{\mathfrak{e}}
\newcommand{\ft}{\mathfrak{t}}
\newcommand{\fr}{\mathfrak{r}}
\newcommand{\fg}{\mathfrak{g}}
\newcommand{\fl}{\mathfrak{l}}
\newcommand{\fa}{\mathfrak{a}}
\newcommand{\fd}{\mathfrak{d}}
\newcommand{\qAff}{\mathsf{qAff}}
\newcommand{\Aff}{\mathsf{Aff}}
\newcommand{\Alg}{\mathsf{Alg}}
\newcommand{\Set}{\mathsf{Set}}
\newcommand{\Mod}{\mathsf{Mod}}
\usepackage{esint}
\renewcommand{\v}{\upsilon}
\newcommand{\vp}{\v_{p}}

% Tikz и графика:


\usepackage{pgfplots}
\usepackage{tikz}
\usetikzlibrary{3d,perspective}
\usetikzlibrary{animations}
\usetikzlibrary{cd}
\usepackage{mathtools}
\pgfplotsset{width=6cm,compat=newest}

\newcommand{\RNum}[1]{\uppercase\expandafter{\romannumeral #1\relax}}




\usepackage{multicol}
\usepackage{pdflscape}
\usepackage{wrapfig}
\begin{document}
	\begin{landscape}

	\pagestyle{empty}
	\begin{multicols}{2}

	\centerline{\bf{Серия 2. Нормальное завтра. }}

	\epigraph{Whenever groups disclose themselves, or could be introduced, simplicity crystallized out of comparative chaos.}{E. T. Bell}

	%\bf{0.} Пусть $\varphi \colon G \to H$~--- гомоморфизм групп. а) Докажите, что $\Ker{\varphi}$~--- подгруппа в $G$. б) Докажите, что $\Im{\varphi}$~--- подгруппа в $H$. в) Докажите, что $\varphi$ инъективен тогда и только тогда, когда $\Ker{\varphi} = \{ e_{G} \}$. 

	\bf{0.} а) Верно ли, что $C_{2} \times C_{4} \cong C_{8}$? б) Опишите условия, при которых  $\Z/m \times \Z/n \cong \Z/nm$ (и докажите изоморфность в этом случае). 


	\bf{1.} Пусть $G$~--- конечная абелева группа, в которой в точности один элемент $f$ порядка 2. Докажите, что 
	\[
		\prod_{g \in G} g = f.
	\]

	\bf{2.} Пусть $G$~--- группа чётного порядка. Докажите, что в ней есть элемент порядка $2$. 

	\bf{3.} Пусть $g \in G$~--- элемент нечётного порядка. Что можно сказать о порядке $g^2$?

	 \bf{4.}  Пусть $N$~--- подгруппа группы $G$, будем говорить, что она \emph{удовлетворяет свойству $(\cH)$}, если $\forall g \in G \ \forall n \in N$ существует такой $n' \in N$, что $gn = n'g$.

	Для подгруппы $H$ и элемента $g$ будем обозначать $g H = \{ gh \ \vert \ h \in H \}$, а для $R \subset G$ будем обозначать $RH = \{ r h \vert \ r \in R, h \in H \}$.

	а) Докажите, что если $N$ это подгруппа, то
	\vspace{-2mm}
	\begin{enumerate}
		\item $NN = N$. 
		\item $N^{-1} = N$. 
		\item Если $\forall g\in G\  \forall n \in N\colon g n g^{-1} \in N$, то $N$ удовлетворяет свойству $(\cH)$.
	\end{enumerate}

	б) Докажите, что если $N$ удовлетворяет свойству $(\cH)$, то
	\vspace{-2mm}
	\begin{enumerate}
		\item $\forall g \ gN = Ng$.
		\item $(gN)(hN) = (gh)N$ и приведите пример, когда это не так если $N$ не удовлетворяет условию $(\cH)$.
	\end{enumerate}

	в) Пусть $\varphi \colon G \to H$~--- гомоморфизм групп, $N = \Ker{\varphi}$. Докажите, что $N$ удовлетворяет свойству $(\cH)$.

	\bf{5.} Неряшливый преподаватель выписал на доску список из девяти целых чисел, образующих группу по умножению по модулю $91$. К сожалению, он забыл выписать одно из чисел и на доске были выписаны лишь числа $1, 9, 16, 22, 53, 74, 79, 81$. Какое число он забыл написать? 

	\begin{definition} 
		\emph{Моноидом} называется множество $M$ с ассоциативной бинарной операцией и нейтральным элементом.   
	\end{definition}

	\bf{6.} Пусть $M_1, M_2$~--- моноиды. Отображение $\varphi \colon M_1 \to M_2$ назовём \emph{хорошим}, если $\forall a, b \in M_1 \ \varphi(ab) = \varphi(a) \varphi(b)$
	  Верно ли, что если $\varphi$~--- хорошее отображение, то $\varphi(e_{M_1}) = e_{M_2}$?

	 \bf{7.}  Пусть $G, H$~--- группы,  $G \cong H \times G$. Можно ли из этого заключиь, что $H$~--- тривиальная группа. (\emph{Подсказска.} Нет! Попробуйте построить контрпример.)

	 \bf{8.} Докажите, что $(\Q, +)$ не может быть представлена в виде произведения двух нетривиальных групп. 


	\columnbreak

	
	\centerline{\bf{Серия 2. Нормальное завтра. }}

	\epigraph{Whenever groups disclose themselves, or could be introduced, simplicity crystallized out of comparative chaos.}{E. T. Bell}

	%\bf{0.} Пусть $\varphi \colon G \to H$~--- гомоморфизм групп. а) Докажите, что $\Ker{\varphi}$~--- подгруппа в $G$. б) Докажите, что $\Im{\varphi}$~--- подгруппа в $H$. в) Докажите, что $\varphi$ инъективен тогда и только тогда, когда $\Ker{\varphi} = \{ e_{G} \}$. 

	\bf{0.} а) Верно ли, что $C_{2} \times C_{4} \cong C_{8}$? б) Опишите условия, при которых  $\Z/m \times \Z/n \cong \Z/nm$ (и докажите изоморфность в этом случае). 


	\bf{1.} Пусть $G$~--- конечная группа, в которой в точности один элемент $f$ порядка 2. Докажите, что 
	\[
		\prod_{g \in G} g = f.
	\]

	\bf{2.} Пусть $G$~--- группа чётного порядка. Докажите, что в ней есть элемент порядка $2$. 

	\bf{3.} Пусть $g \in G$~--- элемент нечётного порядка. Что можно сказать о порядке $g^2$?

	 \bf{4.}  Пусть $N$~--- подгруппа группы $G$, будем говорить, что она \emph{удовлетворяет свойству $(\cH)$}, если $\forall g \in G \ \forall n \in N$ существует такой $n' \in N$, что $gn = n'g$.

	Для подгруппы $H$ и элемента $g$ будем обозначать $g H = \{ gh \ \vert \ h \in H \}$, а для $R \subset G$ будем обозначать $RH = \{ r h \vert \ r \in R, h \in H \}$.

	а) Докажите, что если $N$ это подгруппа, то
	\vspace{-2mm}
	\begin{enumerate}
		\item $NN = N$. 
		\item $N^{-1} = N$. 
		\item Если $\forall g\in G\  \forall n \in N\colon g n g^{-1} \in N$, то $N$ удовлетворяет свойству $(\cH)$.
	\end{enumerate}

	б) Докажите, что если $N$ удовлетворяет свойству $(\cH)$, то
	\vspace{-2mm}
	\begin{enumerate}
		\item $\forall g \ gN = Ng$.
		\item $(gN)(hN) = (gh)N$ и приведите пример, когда это не так если $N$ не удовлетворяет условию $(\cH)$. 
	\end{enumerate}

	в) Пусть $\varphi \colon G \to H$~--- гомоморфизм групп, $N = \Ker{\varphi}$. Докажите, что $N$ удовлетворяет свойству $(\cH)$.

	\bf{5.} Неряшливый преподаватель выписал на доску список из девяти целых чисел, образующих группу по умножению по модулю $91$. К сожалению, он забыл выписать одно из чисел и на доске были выписаны лишь числа $1, 9, 16, 22, 53, 74, 79, 81$. Какое число он забыл написать? 

	\begin{definition} 
		\emph{Моноидом} называется множество $M$ с ассоциативной бинарной операцией и нейтральным элементом.   
	\end{definition}

	\bf{6.} Пусть $M_1, M_2$~--- моноиды. Отображение $\varphi \colon M_1 \to M_2$ назовём \emph{хорошим}, если $\forall a, b \in M_1 \ \varphi(ab) = \varphi(a) \varphi(b)$
	  Верно ли, что если $\varphi$~--- хорошее отображение, то $\varphi(e_{M_1}) = e_{M_2}$?

	 \bf{7.}  Пусть $G, H$~--- группы,  $G \cong H \times G$. Можно ли из этого заключиь, что $H$~--- тривиальная группа. (\emph{Подсказска.} Нет! Попробуйте построить контрпример.)

	 \bf{8.} Докажите, что $(\Q, +)$ не может быть представлена в виде произведения двух нетривиальных групп. 
	

	\end{multicols}
	\end{landscape}


\end{document}