\documentclass[11pt]{extarticle}

\input{preamble}
\usepackage{multicol}
\usepackage{pdflscape}
\usepackage{wrapfig}
\setlength\columnsep{30pt}
\begin{document}
	\begin{landscape}

	\pagestyle{empty}
	\begin{multicols}{2}

	\centerline{\bf{Серия 6. Разгрузочное. }}

	%\epigraph{Свободная группа с двумя образующими~--- это не группа, это просто набор слов.}{Анатолий Моисеевич Вершик}

	\epigraph{Все мы по-разному смотрим на понятие группы. Физики смотрят на группы, как на группы Ли, как на непрерывные группы. Я беседовал с одним очень продвинутым физиком, пытался ему объяснить, что такое свободная группа на двух образующих, очень трудно было!}{Роман Валерьевич Михайлов}

	Напомним, что $\mathrm{Aut}{X^*}$~--- группа автоморфизмов корневого дерева $X^*$, а $S_{X}$~---  симметрическая группа на множестве $X$.

	Пусть $g\colon X^* \to X^*$~--- \bf{эндоморфизм} корневого дерева $X^*$. Рассмотрим поддеревья $v X^*$ и $g(v) X^*$. Тогда $g$ индуцирует мофизм корневых деревьев $g\colon v X^* \to g(v) X^*$ (\bf{нарисуйте картинку!}). 

	Заметим, что корневое поддерево $v X^*$ естественно изоморфно \bf{всему} дереву $X^*$ (вспомните, что изоморфизм задаётся отображением $vw \mapsto w$). Тот же факт $g(v)X^*$. Отождествляя оба поддерева $v X^*$ и $g(v) X^*$ с $X^*$ мы получаем отображение $g\vert_{v} \colon X^* \to X^*$, однозначно определённое формулой 
	\[
		g(vw) = g(v) g\vert_{v}(w).
	\]

	Мы будем называть эндоморфизм $g\vert_{v}$ \emph{сужением} $f$ на вершину $v$.

	\bf{0.} Докажите, что выполняются следующие свойства сужения: а) $g\vert_{v_1 v_2} = g\vert_{v_1}\vert_{v_2}$, б) $(g_1 \cdot g_2)\vert_{v} = g_1\vert_{g_2(v)} \cdot g_2\vert_{v}$. 

	\bf{1.} Постройте изоморфизм 
	\[
		\mathrm{Aut}{X^*} \to S_{X} \wr \mathrm{Aut}{X^*}. 
	\]

	\bf{2.} Пусть $H$~--- группа, действующая на конечном множестве $X$, а $G$~--- произвольная группа. Представьте $H \wr G$ в виде (нетривиального) полупрямого произведения групп. 
	
	 Хоть эта задача и не сложная, в ней оцениваются любые продвижения! 

	 \bf{3.} Опишите группу аффинных преобразований прямой $\R$ (т.е. функция вида $ax + b, \ a, b \in \R$)	, как полупрямое произведение двух групп.
	

	\columnbreak

	\centerline{\bf{Серия 6. Разгрузочное. }}

	%\epigraph{Свободная группа с двумя образующими~--- это не группа, это просто набор слов.}{Анатолий Моисеевич Вершик}

	\epigraph{Все мы по-разному смотрим на понятие группы. Физики смотрят на группы, как на группы Ли, как на непрерывные группы. Я беседовал с одним очень продвинутым физиком, пытался ему объяснить, что такое свободная группа на двух образующих, очень трудно было!}{Роман Валерьевич Михайлов}

	Напомним, что $\mathrm{Aut}{X^*}$~--- группа автоморфизмов корневого дерева $X^*$, а $S_{X}$~---  симметрическая группа на множестве $X$.

	Пусть $g\colon X^* \to X^*$~--- \bf{эндоморфизм} корневого дерева $X^*$. Рассмотрим поддеревья $v X^*$ и $g(v) X^*$. Тогда $g$ индуцирует мофизм корневых деревьев $g\colon v X^* \to g(v) X^*$ (\bf{нарисуйте картинку!}). 

	Заметим, что корневое поддерево $v X^*$ естественно изоморфно \bf{всему} дереву $X^*$ (вспомните, что изоморфизм задаётся отображением $vw \mapsto w$). Тот же факт $g(v)X^*$. Отождествляя оба поддерева $v X^*$ и $g(v) X^*$ с $X^*$ мы получаем отображение $g\vert_{v} \colon X^* \to X^*$, однозначно определённое формулой 
	\[
		g(vw) = g(v) g\vert_{v}(w).
	\]

	Мы будем называть эндоморфизм $g\vert_{v}$ \emph{сужением} $f$ на вершину $v$.

	\bf{0.} Докажите, что выполняются следующие свойства сужения: а) $g\vert_{v_1 v_2} = g\vert_{v_1}\vert_{v_2}$, б) $(g_1 \cdot g_2)\vert_{v} = g_1\vert_{g_2(v)} \cdot g_2\vert_{v}$. 

	\bf{1.} Постройте изоморфизм 
	\[
		\mathrm{Aut}{X^*} \to S_{X} \wr \mathrm{Aut}{X^*}. 
	\]

	\bf{2.} Пусть $H$~--- группа, действующая на конечном множестве $X$, а $G$~--- произвольная группа. Представьте $H \wr G$ в виде (нетривиального) полупрямого произведения групп. 
	
	 Хоть эта задача и не сложная, в ней оцениваются любые продвижения! 

	 \bf{3.} Опишите группу аффинных преобразований прямой $\R$ (т.е. функция вида $ax + b, \ a, b \in \R$), как полупрямое произведение двух групп.
	

 	

	\end{multicols}
	\end{landscape}


\end{document}