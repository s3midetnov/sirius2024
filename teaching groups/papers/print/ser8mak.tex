\documentclass[12pt]{extarticle}

\input{preamble}
\usepackage{multicol}
\usepackage{pdflscape}
\usepackage{wrapfig}
\setlength\columnsep{30pt}
\begin{document}
	\begin{landscape}

	\pagestyle{empty}
	\begin{multicols}{2}

	
	\centerline{\bf{Серия 8.  Растём над собой\ldots }}

	\epigraph{Свободная группа с двумя образующими~--- это не группа, это просто набор слов.}{Анатолий Моисеевич Вершик}

	\bf{0.} Чему изоморфна группа $\langle a, b, c \ \vert \ a^{5} = 1, \ b^{17} = 1, \ c^{239} = 1, \ [a, b] = 1, \ [a, c] = 1 , \ [b, c] = 1 \rangle$.

	\bf{1.} В группе $G$ выполнены соотношения $a^2 = b^5 = (ab)^4$ и $(a b^{-2} a b^2)^2 = a^2$. Докажите, что $(ba)^4 = 1$. 

	\bf{2.}  Мама отправила Васю в магазин <<Мир теории групп>>, чтоб он принёс полезную в быту группу, но написала в списке не её название, а задание образующими и соотношениями:
	\[
		G = \langle \eta_1, \ldots, \eta_{n - 1} \ \vert \ \eta_j^2 = 1, \ (\eta_j \eta_{j + 1})^3, \ \eta_j \eta_{\ell} = \eta_{\ell} \eta_{j} \ \forall \ j, \ell \colon |j - \ell| \ge 1 \rangle. 
	\]

	Помогите Васе понять, какую группу ему надо купить в магазине. 

	\bf{3.} Пусть $G$~--- транзитивная на уровнях подгруппа в группе автоморфизмов $\mathrm{Aut}(X^*)$, а $w \in X^*$. Докажите, что если $h \in G[w]$ и $g(w) \neq w$, то $[h, g] \neq 1$. 

	\bf{4.} Рассмотрим в группе Григорчука $\mathbf{G}$ следующее правивло переписывания 
	\[
		\tau\colon a \mapsto a c a, \ c \mapsto c d, \ d \mapsto c.
	\]
	Докажите, что $\tau^i(ad)^4 = 1$
	
	 	
	\bf{5.} Рассмотрим группу $\Z = \langle r, s \rangle$, где $0 < r < s$. Докажите, что для достаточно больших $n$ сферическая функция роста будет иметь вид  $s_{\Z}(n) = 2s$.  

	\bf{6.} Пусть $G$~--- произвольная конечнопорожденная группа. Докажите, что для её шаровой функции роста выполнено неравенство
	\[
	 	b(n + m) \le b(n)b(m).
	 \] 
	 б) выведите из этого, что $\lim\limits_{n \to \infty} (b(n))^{1/n}$ существует и конечен.

	  \bf{7.} Заведём на множестве функций $f\colon \N \to \R_{+}$ такое отношение эквивалентности 
	 \[
	 	f \approx g \Leftrightarrow  f \preceq g \text{ и } g \preceq f,
	 \]
	 где $f \leqslant g$, если $\exists A \ge 1\colon f(n) \le A \cdot g(An)$ для достаточно больших $n$. Убедитесь, что это в самом деле отношение эквивалетности. 

	\columnbreak

	\centerline{\bf{Серия 8.  Растём над собой\ldots }}

	\epigraph{Свободная группа с двумя образующими~--- это не группа, это просто набор слов.}{Анатолий Моисеевич Вершик}

	\bf{0.} Чему изоморфна группа $\langle a, b, c \ \vert \ a^{5} = 1, \ b^{17} = 1, \ c^{239} = 1, \ [a, b] = 1, \ [a, c] = 1 , \ [b, c] = 1 \rangle$.

	\bf{1.} В группе $G$ выполнены соотношения $a^2 = b^5 = (ab)^4$ и $(a b^{-2} a b^2)^2 = a^2$. Докажите, что $(ba)^4 = 1$. 

	\bf{2.}  Мама отправила Васю в магазин <<Мир теории групп>>, чтоб он принёс полезную в быту группу, но написала в списке не её название, а задание образующими и соотношениями:
	\[
		G = \langle \eta_1, \ldots, \eta_{n - 1} \ \vert \ \eta_j^2 = 1, \ (\eta_j \eta_{j + 1})^3, \ \eta_j \eta_{\ell} = \eta_{\ell} \eta_{j} \ \forall \ j, \ell \colon |j - \ell| \ge 1 \rangle. 
	\]

	Помогите Васе понять, какую группу ему надо купить в магазине. 

	\bf{3.} Пусть $G$~--- транзитивная на уровнях подгруппа в группе автоморфизмов $\mathrm{Aut}(X^*)$, а $w \in X^*$. Докажите, что если $h \in G[w]$ и $g(w) \neq w$, то $[h, g] \neq 1$. 

	\bf{4.} Рассмотрим в группе Григорчука $\mathbf{G}$ следующее правивло переписывания 
	\[
		\tau\colon a \mapsto a c a, \ c \mapsto c d, \ d \mapsto c.
	\]
	Докажите, что $\tau^i(ad)^4 = 1$
	
	 	
	\bf{5.} Рассмотрим группу $\Z = \langle r, s \rangle$, где $0 < r < s$. Докажите, что для достаточно больших $n$ сферическая функция роста будет иметь вид  $s_{\Z}(n) = 2s$.  

	\bf{6.} Пусть $G$~--- произвольная конечнопорожденная группа. Докажите, что для её шаровой функции роста выполнено неравенство
	\[
	 	b(n + m) \le b(n)b(m).
	 \] 
	 б) выведите из этого, что $\lim\limits_{n \to \infty} (b(n))^{1/n}$ существует и конечен.

	  \bf{7.} Заведём на множестве функций $f\colon \N \to \R_{+}$ такое отношение эквивалентности 
	 \[
	 	f \approx g \Leftrightarrow  f \preceq g \text{ и } g \preceq f,
	 \]
	 где $f \leqslant g$, если $\exists A \ge 1\colon f(n) \le A \cdot g(An)$ для достаточно больших $n$. Убедитесь, что это в самом деле отношение эквивалетности. 

	\end{multicols}
	\end{landscape}



\end{document}