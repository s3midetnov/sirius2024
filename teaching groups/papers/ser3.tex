\documentclass[12pt]{article}

\input{print/preamble}
\begin{document}

	\centerline{\bf{Серия 3. Действию время, а потехе час. }}

	\epigraph{Двадцать две основные буквы: Бог их нарисовал, высек в камне, соединил, взвесил, \emph{переставил} и создал из них все, что есть,~--- и все, что будет.}{Сефер Йецира}

	\bf{0.} Докажите, что $G \times H \cong H \times G$.

	\bf{1.} Пусть $A, B$~--- абелевы группы. Обозначим за $\mathrm{Hom}(A, B)$ множество гомоморфизмов из $A$ в $B$. Задайте на $\mathrm{Hom}(A, B)$ структуру абелевой группы. 

	\bf{2.} Докажите формулу для числа сочетаний $\binom{n}{k}$ при помощи действия группы на множестве. 

	\bf{3.} а) Докажите, что $\sum_{\pi \in S_n} |\mathrm{fix}(\pi)| = n!$. б) Пусть группа $G$ транзитивно действует на множестве $X$. Каково среднее значение числа неподвижных точек элементов по всей $G$, то есть 
	\[
		\frac{1}{|G|}\sum_{g \in G} |\mathrm{fix}(g)|. 
	\]


	\bf{4.} а) Мальчик Вася нарисовал на бесконечном листе бумаги такой концептуальный рисунок: 
	\[
		\ldots \ \Gamma \ \Gamma \ \Gamma \ \Gamma \ \Gamma \ \Gamma \ \Gamma \ \Gamma \ \Gamma \ \Gamma \ldots
	\]
	Найдите группу симметрий этого рисунка. 

	б) то же самое, но для рисунка 

	\[
		\ldots \ \mathrm{D} \ \mathrm{D} \ \mathrm{D} \ \mathrm{D} \ \mathrm{D}\ \mathrm{D} \ \mathrm{D} \ \mathrm{D} \ \mathrm{D} \ \mathrm{D}\ldots
	\]

	\noindent\bf{Комментарий.} в данной задаче мы ищем \emph{только те симметрии, которые переводят соседние буквые в соседние}. И, отметим, что буквы стоят на \emph{одинаковом} расстоянии. 

	\begin{definition} 
		Симметрической группой $S_{X}$ на множестве $X$ называется группа биекций $X \to X$.
	\end{definition}

	\bf{5.} Докажите, что любая группа $G$ реализуется, как подгруппа в некоторой симметрической группе (т.е. симметрической группе какого-то множество). \emph{Подсказка.} Если у нас есть инъективный гомоморфизм $\varphi\colon G \to H$ по очевидным причинам $\mathrm{Im}{\varphi} \cong G$ и мы можем отождествить образ с самой группой $G$. 

	\begin{definition} 
		\emph{Дробно-линейным} преобразованием называется функция $f\colon \HH^2 \to \HH^2$ вида 
		\[
			f(z) = \frac{az + b}{cz + d}, \quad a, b, c, d \in \R,\  ad - bc = 1.
		\]
		Здесь $\HH^2 = \{ z \in \C \ \vert \ \Im{z} > 0 \}$. 
	\end{definition}


	\bf{6.} Докажите, что дробно-линейные преобразования образуют группу относительно композиции.

	
\end{document}