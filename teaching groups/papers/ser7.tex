\documentclass[12pt]{article}

\input{print/preamble}
\begin{document}

	\centerline{\bf{Серия 7. Папа Карло тоже действовал на дереве\ldots  }}

	\epigraph{Целый новый ряд мыслей безнадежных, но грустно-приятных в связи с этим дубом возник в душе князя Андрея. Во время этого путешествия он как будто вновь обдумал всю свою жизнь и пришел к тому же прежнему, успокоительному и безнадежному, заключению, что ему начинать ничего было не надо, что он должен доживать свою жизнь, не делая зла, не тревожась и ничего не желая.}{Л.Н. Толстой, <<Война и мир>>}

	В этой сери задач $\mathbf{G}$~--- группа Григорчука, а $a, b, c, d$~--- как в лекции. 

	\bf{1.} а) Проверьте, что $a, b, c, d$ имеют порядок 2, коммутируют друг с другом и удовлетворяют групповому тождеству  $b \cdot c \cdot d = 1$. б) Выведите отсюда, что $\langle b, c, d \rangle \cong \Z_{2}^2$. в) Докажите, что $\mathbf{G} = \langle a, b, c, d \rangle$ 3-порожденная. 
	
	\bf{2.} а) Проверьте, что в группе $\mathbf{G}$ выполняются соотношения 
	\[
		(ad)^4 = (ac)^8 = (ab)^{16} = 1.
	\]
	б) Выведите отсюда, что подгруппы $\langle a, b \rangle$, $\langle a, c \rangle$, $\langle a, d \rangle$ группы $\mathbf{G}$ конечны. 

	\bf{3.} Вася переписывает элементы группы $\mathbf{G}$ по следующимс правилам
	\[
		\xi\colon a \mapsto aba, \ b \mapsto d, \ c \mapsto b, \ d \mapsto c.
	\]
	а) Помогите Васе построить последовательность элементов $\mathbf{G}$ такую, что $x_1 = a$ и $\forall i \ge 1 \ x_{i + 1} = \xi(x_{i})$. б) Докажите, что все элементы $x_i$ различны. в) Выведите из этого, что  $\mathbf{G}$ бесконечна.

\end{document}